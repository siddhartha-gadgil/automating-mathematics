+++ title = ``SAT Solving'' date = 2021-05-18T08:53:42+05:30 draft =
false tags = {[}{]} categories = {[}{]} css = ``sat.css'' +++

To better understand the \texttt{SAT} (i.e., \emph{boolean
satisfiability}) problem and \texttt{SAT} solvers, I decided to
implement a basic one. I was pleasantly surprised that wikipedia has
enough details to implement the so called \textbf{DPLL} algorithm quite
easily, with even some improvements described in wikipedia. Even better,
in the case when there was no solution, the same algorithm gives a proof
that there is no solution. The proof that there is no solution was based
on \emph{resolution} due to Davis-Putnam --- so the algorithm gives as a
bonus a proof that resolution is refutation complete for propositional
calculus.

The code corresponding to this blog is in the \texttt{fol} project (and
directory) of my
\href{https://github.com/siddhartha-gadgil/ProvingGround}{ProvingGround}
repository.

\hypertarget{the-sat-boolean-satisfiability-problem}{%
\subsection{\texorpdfstring{The \texttt{SAT} (Boolean satisfiability)
problem}{The SAT (Boolean satisfiability) problem}}\label{the-sat-boolean-satisfiability-problem}}

Suppose we are given finitely many variables \(P\), \(Q\), \ldots{}
which are \emph{boolean}, i.e.~represent whether a statement is true or
false, and finitely many relations among them. The relations are logical
statements built from the variables using logical operators such as
\(\neg\) (\emph{not}), \(\vee\)(\emph{or}), \(\wedge\)(\emph{and}),
\(\Rightarrow\)(\emph{implies}) and \(\Leftrightarrow\)
(\emph{equivalent}). The \texttt{SAT} (\emph{boolean satisfiability})
problem asks whether we can assign truth values to these (i.e., declare
each of \(P\), \(Q\) , \ldots{} to be \texttt{true} or \texttt{false})
so that all the relations are satisfied.

In the sequel, we shall use the standard logic terminology. Logical
statements are called \emph{Propositions} (these can be either true or
false, unlike the standard meaning of the word proposition in
mathematics). Statements built from logical variables by logical
operators are called \emph{formulas}. All the variables we consider are
boolean. Note that a boolean variable is itself a formula.

\hypertarget{cnf}{%
\subsection{CNF}\label{cnf}}

If \(P\) is a (boolean) variable, the formulas \(P\) and \(\neg P\) are
called \emph{literals}. A \emph{clause} is a formula of the form
\(l_1\vee l_2\vee\dots\vee l_n\) with each \(l_i\) a literal. A formula
in \textbf{CNF} is a formula of the form
\(C_1\wedge C_2\wedge\dots\wedge C_m\) such that each \(C_j\) is a
clause. Observe that \(C_1\wedge C_2\wedge\dots\wedge C_m\) is satisfied
if and only if all the formulas \(C_j\) are satisfied. Hence we can (and
will) view a formula in \(CNF\) as a collection of clauses, all of which
are required to be satisfied.

Any formula can be rewritten as a CNF formula to which it is equivalent.
Namely, we first rewrite all instances of \(A\Rightarrow B\) as
\(\neg B\vee A\) and all instances of \(A\Leftrightarrow B\) as
\((\neg A\wedge \neg B)\vee (A\wedge B)\) to eliminate operators other
than \(\vee\), \(\wedge\) and \(\neg\). Next we use
\(\neg(A \vee B) = \neg A \wedge \neg B\) and
\(\neg(A \wedge B) = \neg A \vee \neg B\) recursively to rewrite the
formula as combinations of literals using \(\vee\) and \(\wedge\) only.
Finally, using the distributivity property
\(A \vee (B\wedge C)= (A\vee B)\wedge (A \vee C)\) recursively we get a
formula in CNF.

\hypertarget{tautologies-and-contradictions}{%
\subsubsection{Tautologies and
Contradictions}\label{tautologies-and-contradictions}}

Since \(l\vee l = l\), we can assume that no literal appears more than
once in a clause. Further, if a clause contains both \(P\) and
\(\neg P\) for some variable \(P\), then this clause is always true,
i.e., is a \emph{tautology}. We can drop such clauses. We henceforth
assume both these simplifications have been made.

On the other hand, an empty clause can never be satisfied (as
\emph{some} literal in the clause must be true for it to be satisfied).
Thus, an empty clause is a \emph{contradiction}, and we denote it by
\(\bot\). In particular, the formula in CNF has an empty clause, then it
cannot be satisfied. This will rarely happen, but we shall see how to
deduce new clauses in such a way as to get an empty clause for
unsatisfiable formulas.

\hypertarget{resolution}{%
\subsubsection{Resolution}\label{resolution}}

Given two clauses that are of the form
\(C = P\vee l_1\vee l_2\vee \dots\vee l_n\) and
\(C' = \neg P\vee l'_1\vee l'_2\vee \dots\vee l'_{n'}\) (after possibly
reordering literals), we can deduce the clause
\(l_1\vee l_2\vee \dots\vee l_n\vee l'_1\vee l'_2\vee \dots\vee l'_{n'}\).
Namely, if \(P\) is false then as \(C\) is true we must have
\(l_1\vee l_2\vee \dots\vee l_n\). On the other hand if \(P\) is true,
as \(C'\) is true we must have \(l'_1\vee l'_2\vee \dots\vee l'_{n'}\).
Either way,
\(l_1\vee l_2\vee \dots\vee l_n\vee l'_1\vee l'_2\vee \dots\vee l'_{n'}\)
holds.

The clause
\(l_1\vee l_2\vee \dots\vee l_n\vee l'_1\vee l'_2\vee \dots\vee l'_{n'}\)
is said to be obtained by \emph{resolution} from \(C\) and \(C'\). A
theorem of Martin Davis and Hilary Putnam says that resolution is
\emph{refutation complete} not just for \texttt{SAT} but in a more
general context, namely first-order logic. This means that if a
collection of clauses is not satisfiable, then we can deduce the empty
clause by (repeated) resolution starting with the given clauses. Note
that the final step will be resolving (singleton) clauses of the form
\(P\) and \(\neg P\) to get the empty clause \(\bot\).

The algorithm I implemented gives either a solution to a collection of
clauses or a deduction using resolution of the empty clause \(\bot\).
Thus, we get in particular a proof that resolution is refutation
complete in the context of \texttt{SAT}.

\hypertarget{an-example-the-n-queens-problem}{%
\subsection{An example: the N-Queens
problem}\label{an-example-the-n-queens-problem}}

Before we sketch our algorithm, we consider a class of examples -- the
N-Queens problem. This asks whether we can place \(N\) queens on an
\(N\times N\) chessboard with no two queens attacking each other.

To formulate this in terms of \texttt{SAT}, we consider a collection of
\(N^2\) boolean variables \texttt{QueenAt(i,\ j)} for
\(0 \leq i, j < N\) representing whether a queen is present at the
corresponding square on the grid (as is common with programs the indices
begin at \(0\)). As is usual, instead of using (the clumsy) equation for
their being (at least) \(N\) queens on the board, we use \(N\) equations
saying that each row has a queen. We also have a bunch of equation for
queens not attacking each other horizontally, vertically and along
diagonals. These equations are generated (programmatically) as clauses.

\hypertarget{solution-to-8-queens}{%
\subsubsection{Solution to 8-Queens}\label{solution-to-8-queens}}

The classical 8-Queens problem is solved instantly. A little scripting
lets us display the solution in a table (using a unicode character for
the queen).

\begin{longtable}[]{@{}llllllll@{}}
\toprule
\endhead
& & & ♕ & & & &\tabularnewline
& & & & & & & ♕\tabularnewline
♕ & & & & & & &\tabularnewline
& & ♕ & & & & &\tabularnewline
& & & & & ♕ & &\tabularnewline
& ♕ & & & & & &\tabularnewline
& & & & & & ♕ &\tabularnewline
& & & & ♕ & & &\tabularnewline
\bottomrule
\end{longtable}

\hypertarget{no-solution-for-3-queens}{%
\subsubsection{No solution for
3-Queens}\label{no-solution-for-3-queens}}

On the other hand, the 3-Queens problem has no solutions. Indeed, our
program gives a proof of this by deducing a contradiction using
resolution, starting with the assumptions. A little scripting lets us
write this solution as nested lists, with resolutions written in terms
of the final clause being deduced from the given clauses. This gives a
tree-like structure, with the root a contradiction and the leaves the
given assumptions.

\begin{itemize}
\tightlist
\item
  \emph{Contradiction} \textbf{from} (¬QueenAt(1,0)) \textbf{and}
  (QueenAt(1,0)), \textbf{using}

  \begin{itemize}
  \tightlist
  \item
    (¬QueenAt(1,0)) \textbf{from} (¬QueenAt(1,0) ∨ QueenAt(0,0))
    \textbf{and} (¬QueenAt(0,0) ∨ ¬QueenAt(1,0)), \textbf{using}

    \begin{itemize}
    \tightlist
    \item
      (¬QueenAt(1,0) ∨ QueenAt(0,0)) \textbf{from} (¬QueenAt(1,0) ∨
      ¬QueenAt(0,2)) \textbf{and} (¬QueenAt(1,0) ∨ QueenAt(0,0) ∨
      QueenAt(0,2)), \textbf{using}

      \begin{itemize}
      \tightlist
      \item
        (¬QueenAt(1,0) ∨ ¬QueenAt(0,2)) \textbf{from} (¬QueenAt(1,0) ∨
        QueenAt(2,2)) \textbf{and} (¬QueenAt(0,2) ∨ ¬QueenAt(2,2)),
        \textbf{using}

        \begin{itemize}
        \tightlist
        \item
          (¬QueenAt(1,0) ∨ QueenAt(2,2)) \textbf{from} (¬QueenAt(2,1) ∨
          ¬QueenAt(1,0)) \textbf{and} (¬QueenAt(1,0) ∨ QueenAt(2,1) ∨
          QueenAt(2,2)), \textbf{using}

          \begin{itemize}
          \tightlist
          \item
            (¬QueenAt(2,1) ∨ ¬QueenAt(1,0)) by assumption
          \item
            (¬QueenAt(1,0) ∨ QueenAt(2,1) ∨ QueenAt(2,2)) \textbf{from}
            (¬QueenAt(2,0) ∨ ¬QueenAt(1,0)) \textbf{and} (QueenAt(2,1) ∨
            QueenAt(2,2) ∨ QueenAt(2,0)), \textbf{using}

            \begin{itemize}
            \tightlist
            \item
              (¬QueenAt(2,0) ∨ ¬QueenAt(1,0)) by assumption
            \item
              (QueenAt(2,1) ∨ QueenAt(2,2) ∨ QueenAt(2,0)) by assumption
            \end{itemize}
          \end{itemize}
        \item
          (¬QueenAt(0,2) ∨ ¬QueenAt(2,2)) by assumption
        \end{itemize}
      \item
        (¬QueenAt(1,0) ∨ QueenAt(0,0) ∨ QueenAt(0,2)) \textbf{from}
        (¬QueenAt(0,1) ∨ ¬QueenAt(1,0)) \textbf{and} (QueenAt(0,0) ∨
        QueenAt(0,2) ∨ QueenAt(0,1)), \textbf{using}

        \begin{itemize}
        \tightlist
        \item
          (¬QueenAt(0,1) ∨ ¬QueenAt(1,0)) by assumption
        \item
          (QueenAt(0,0) ∨ QueenAt(0,2) ∨ QueenAt(0,1)) by assumption
        \end{itemize}
      \end{itemize}
    \item
      (¬QueenAt(0,0) ∨ ¬QueenAt(1,0)) by assumption
    \end{itemize}
  \item
    (QueenAt(1,0)) \textbf{from} (QueenAt(1,0) ∨ QueenAt(2,1))
    \textbf{and} (¬QueenAt(2,1) ∨ QueenAt(1,0)), \textbf{using}

    \begin{itemize}
    \tightlist
    \item
      (QueenAt(1,0) ∨ QueenAt(2,1)) \textbf{from} (QueenAt(1,2) ∨
      QueenAt(1,0) ∨ QueenAt(2,1)) \textbf{and} (¬QueenAt(1,2) ∨
      QueenAt(2,1)), \textbf{using}

      \begin{itemize}
      \tightlist
      \item
        (QueenAt(1,2) ∨ QueenAt(1,0) ∨ QueenAt(2,1)) \textbf{from}
        (QueenAt(1,2) ∨ QueenAt(1,0) ∨ ¬QueenAt(2,0)) \textbf{and}
        (QueenAt(1,2) ∨ QueenAt(1,0) ∨ QueenAt(2,0) ∨ QueenAt(2,1)),
        \textbf{using}

        \begin{itemize}
        \tightlist
        \item
          (QueenAt(1,2) ∨ QueenAt(1,0) ∨ ¬QueenAt(2,0)) \textbf{from}
          (QueenAt(1,1) ∨ QueenAt(1,2) ∨ QueenAt(1,0)) \textbf{and}
          (¬QueenAt(2,0) ∨ ¬QueenAt(1,1)), \textbf{using}

          \begin{itemize}
          \tightlist
          \item
            (QueenAt(1,1) ∨ QueenAt(1,2) ∨ QueenAt(1,0)) by assumption
          \item
            (¬QueenAt(2,0) ∨ ¬QueenAt(1,1)) by assumption
          \end{itemize}
        \item
          (QueenAt(1,2) ∨ QueenAt(1,0) ∨ QueenAt(2,0) ∨ QueenAt(2,1))
          \textbf{from} (QueenAt(1,2) ∨ QueenAt(1,0) ∨ ¬QueenAt(2,2))
          \textbf{and} (QueenAt(2,0) ∨ QueenAt(2,2) ∨ QueenAt(2,1)),
          \textbf{using}

          \begin{itemize}
          \tightlist
          \item
            (QueenAt(1,2) ∨ QueenAt(1,0) ∨ ¬QueenAt(2,2)) \textbf{from}
            (QueenAt(1,1) ∨ QueenAt(1,2) ∨ QueenAt(1,0)) \textbf{and}
            (¬QueenAt(2,2) ∨ ¬QueenAt(1,1)), \textbf{using}

            \begin{itemize}
            \tightlist
            \item
              (QueenAt(1,1) ∨ QueenAt(1,2) ∨ QueenAt(1,0)) by assumption
            \item
              (¬QueenAt(2,2) ∨ ¬QueenAt(1,1)) by assumption
            \end{itemize}
          \item
            (QueenAt(2,0) ∨ QueenAt(2,2) ∨ QueenAt(2,1)) by assumption
          \end{itemize}
        \end{itemize}
      \item
        (¬QueenAt(1,2) ∨ QueenAt(2,1)) \textbf{from} (¬QueenAt(1,2) ∨
        QueenAt(2,1) ∨ QueenAt(0,1)) \textbf{and} (¬QueenAt(0,1) ∨
        ¬QueenAt(1,2)), \textbf{using}

        \begin{itemize}
        \tightlist
        \item
          (¬QueenAt(1,2) ∨ QueenAt(2,1) ∨ QueenAt(0,1)) \textbf{from}
          (¬QueenAt(1,2) ∨ QueenAt(2,1) ∨ ¬QueenAt(0,0)) \textbf{and}
          (¬QueenAt(1,2) ∨ QueenAt(0,0) ∨ QueenAt(0,1)), \textbf{using}

          \begin{itemize}
          \tightlist
          \item
            (¬QueenAt(1,2) ∨ QueenAt(2,1) ∨ ¬QueenAt(0,0)) \textbf{from}
            (¬QueenAt(1,2) ∨ QueenAt(2,0) ∨ QueenAt(2,1)) \textbf{and}
            (¬QueenAt(0,0) ∨ ¬QueenAt(2,0)), \textbf{using}

            \begin{itemize}
            \tightlist
            \item
              (¬QueenAt(1,2) ∨ QueenAt(2,0) ∨ QueenAt(2,1))
              \textbf{from} (¬QueenAt(2,2) ∨ ¬QueenAt(1,2)) \textbf{and}
              (QueenAt(2,0) ∨ QueenAt(2,2) ∨ QueenAt(2,1)),
              \textbf{using}

              \begin{itemize}
              \tightlist
              \item
                (¬QueenAt(2,2) ∨ ¬QueenAt(1,2)) by assumption
              \item
                (QueenAt(2,0) ∨ QueenAt(2,2) ∨ QueenAt(2,1)) by
                assumption
              \end{itemize}
            \item
              (¬QueenAt(0,0) ∨ ¬QueenAt(2,0)) by assumption
            \end{itemize}
          \item
            (¬QueenAt(1,2) ∨ QueenAt(0,0) ∨ QueenAt(0,1)) \textbf{from}
            (¬QueenAt(0,2) ∨ ¬QueenAt(1,2)) \textbf{and} (QueenAt(0,0) ∨
            QueenAt(0,1) ∨ QueenAt(0,2)), \textbf{using}

            \begin{itemize}
            \tightlist
            \item
              (¬QueenAt(0,2) ∨ ¬QueenAt(1,2)) by assumption
            \item
              (QueenAt(0,0) ∨ QueenAt(0,1) ∨ QueenAt(0,2)) by assumption
            \end{itemize}
          \end{itemize}
        \item
          (¬QueenAt(0,1) ∨ ¬QueenAt(1,2)) by assumption
        \end{itemize}
      \end{itemize}
    \item
      (¬QueenAt(2,1) ∨ QueenAt(1,0)) \textbf{from} (¬QueenAt(2,1) ∨
      QueenAt(1,1) ∨ QueenAt(1,0)) \textbf{and} (¬QueenAt(1,1) ∨
      ¬QueenAt(2,1)), \textbf{using}

      \begin{itemize}
      \tightlist
      \item
        (¬QueenAt(2,1) ∨ QueenAt(1,1) ∨ QueenAt(1,0)) \textbf{from}
        (¬QueenAt(1,2) ∨ ¬QueenAt(2,1)) \textbf{and} (QueenAt(1,1) ∨
        QueenAt(1,2) ∨ QueenAt(1,0)), \textbf{using}

        \begin{itemize}
        \tightlist
        \item
          (¬QueenAt(1,2) ∨ ¬QueenAt(2,1)) by assumption
        \item
          (QueenAt(1,1) ∨ QueenAt(1,2) ∨ QueenAt(1,0)) by assumption
        \end{itemize}
      \item
        (¬QueenAt(1,1) ∨ ¬QueenAt(2,1)) by assumption
      \end{itemize}
    \end{itemize}
  \end{itemize}
\end{itemize}

\hypertarget{solving-sat}{%
\subsection{Solving SAT}\label{solving-sat}}

I now describe the algorithm I implemented to solve \texttt{SAT} for a
collection of clauses, giving either an assignment of truth values that
satisfies all the clauses or a contradiction using resolution starting
with the given clauses. We shall see this in stages.

A \texttt{SAT} problem is specified by a finite set \(V\) of Boolean
variables and a finite set \(E\) of clauses in these variables --- we
refer to this as \emph{the \texttt{SAT} problem \(SAT(V, E)\)}. We also
assume that \(E\) does not contain tautologies, and none of the clauses
contain a literal more than once. We also regard clauses as equal if
they become so after reordering.

\hypertarget{recursive-solving-the-dp-algorithm}{%
\subsubsection{Recursive solving: the DP
algorithm}\label{recursive-solving-the-dp-algorithm}}

Assume that we are given sets \(V\) and \(E\) as above. We pick a
variable \(P\in V\) and look for solutions first when \(P\) is assigned
the value true and then when it is assigned false (in my code I actually
randomized the order of the branches). For solutions where \(P\) is
true:

\begin{enumerate}
\def\labelenumi{\arabic{enumi}.}
\tightlist
\item
  Any clause containing \(P\) is true, so can be dropped.
\item
  Any clause of the form \(\neg P\vee l_1\vee\dots\vee l_n\) (up to
  reordering) is true if and only the clause \(C =l_1\vee\dots\vee l_n\)
  obtained by dropping \(\neg P\) is true. We can thus replace
  \(\neg P \vee C := \neg P\vee l_1\vee\dots\vee l_n\) by \(C\) when
  considering solutions with \(P\) true.
\item
  Finally, clauses \(C\) containing neither \(P\) nor \(\neg P\) are
  unaffected, and should be satisfied even after assigning \(P\).
\end{enumerate}

Thus, we get a new set of clauses, which we denote \(\rho(E, P)\), in
the variables \(V\setminus\{P\}\), giving the \emph{restricted}
\texttt{SAT} problem \(SAT(V\setminus\{P\}, \rho(E, P))\). If the
restricted problem has a solution then we get a solution to the original
problem \(SAT(V, E)\) by assigning true to \(P\). Otherwise we look for
a solution where \(P\) is false, once more obtaining a restricted
\texttt{SAT} problem \(SAT(V\setminus\{P\}, \rho(E, \neg P))\) with
fewer variables (with \(\rho(E, \neg P))\) defined analogous to
\(\rho(E, P))\)). If the latter problem has a solution so does the
original one \(SAT(V, E)\). If not, we know both the restricted problems
have no solutions. Since \(P\) must be true or false, the original
problem \(SAT(V, E)\) has no solution.

Thus, we can reduce the solution of a \texttt{SAT} problem to the
solution of \texttt{SAT} problems with fewer variables. This gives a
recursive algorithm once we solve in the base case, which is the case
with no variables. But in this case the only clause is the empty clause.
Thus there are only two \texttt{SAT} problems.

\begin{itemize}
\tightlist
\item
  \(E = \phi\). Here every clause is satisfied, so there is a solution.
\item
  \(E = \{\bot\}\), i.e., a singleton consisting of the empty clause.
  This is not satisfiable as the empty clause cannot be satisfied.
\end{itemize}

To summarize, we have a recursive algorithm by reducing to a simpler
case, namely with fewer variables, in case the set of variables is
non-empty, and (easily) solving in the case where the set of variables
is empty.

\hypertarget{dpll-algorithm}{%
\subsubsection{DPLL algorithm}\label{dpll-algorithm}}

There are two ways in which the DP algorithm can be improved, giving the
DPLL algorithm. Firstly, if some clause is a \emph{unit literal}, i.e.,
a literal \(l\) with \(l = P\) or \(l = \neg P\), then \(P\) must be
assigned the value that makes \(l\) true. On doing this, all clauses
containing \(l\) are true and can be dropped, while a clause of the form
\(\neg l\vee C\) can be replaced by the clause \(C\) obtained by
deleting \(\neg l\), i.e., we replace \(E\) by \(\rho(E, l)\). On making
such simplifications, new units may be created and this process
repeated. For instance, if each clause has length 2 (so called
\texttt{2-SAT}), this clearly gives a fast algorithm.

A second improvement is to use \emph{pure literals}, which are literals
\(l\) so that \(\neg l\) is not present in any clause. Then the
\texttt{SAT} problem has a solution if and only if it has a solution
with \(l\) true. Hence we can assign the value of the variable \(P\)
with \(l = P\) or \(l = \neg P\) to make \(l\) true, and drop all the
clauses containing \(l\).

I have implemented this without further heuristics, except for some use
of a \emph{conflict driven} approach to avoid full back-tracking, based
on keeping track of proofs. I next sketch how we keep track of proofs
(the algorithmic improvement will be evident).

\hypertarget{lifting-proofs}{%
\subsubsection{Lifting proofs}\label{lifting-proofs}}

Suppose the given \texttt{SAT} problem \(SAT(V, E)\) has no solution. We
can enhance our algorithm to in this case give a proof using resolution
of a contradiction starting with the given clauses \(E\). I sketch this
here. The main step is the \emph{lifting} of a proof from a restricted
problem. We shall denote the set of clauses deduced by resolution from a
set of clauses \(E\) by \(D(E)\). Thus, \(D(E)\supset E\), and is the
smallest set containing \(E\) that is closed under resolution.

To start with consider the base case where there is no variable. Here if
the \texttt{SAT} problem is not satisfiable, we must have the empty
clause in \(E\). Thus a given clause is itself a contradiction.

Now consider the general case with \(n\) variables, and assume that our
algorithm gives either a solution or deduces a contradiction using
resolution whenever we have fewer than \(n\) variables. Pick a variable
\(P\) and assign that \(P\) is true. As above, we get a restricted
problem \(SAT(V\setminus\{P\}, \rho(E, P))\) not involving the variable
\(P\).

Suppose the original problem \(SAT(V, E)\) has no solution, then the
restricted problem \(SAT(V\setminus\{P\}, \rho(E, P))\) does not either.
By the induction hypothesis, using resolution we can deduce a
contradiction starting with \(\rho(E, P)\), i.e., we have
\(\bot\in D(\rho(E, P))\). We cannot in general conclude from this that
\(\bot\in D(E)\). Nevertheless we can lift the proof of a contradiction
to something useful.

Observe that, by the construction of \(\rho(E, P)\), if
\(C \in \rho(E, P)\) then either \(C \in E\) or \(\neg P\vee C\in E\)
(if both hold we choose the first case in the sequel for efficiency). We
claim that an analogous statement holds for clauses deduced by
resolution. Namely, if \(C \in D(\rho(E, P))\) then either
\(C \in D(E)\) or \(\neg P\vee C\in D(E)\).

This claim can be proved by induction on the number of steps in the
deduction using resolution. For the case with \(0\) steps, we just get
elements of \(\rho(E, P)\), for which we have the claim by the above
observations as \(E\subset D(E)\).

Next, if \(C \in D(\rho(E, P))\) is deduced in \(n > 0\) steps, then
\(C\) can be deduced from clauses \(C_1\) and \(C_2\) by resolution, and
the clauses \(C_i\) can be deduced from \(\rho(E, p)\) in fewer than
\(n\) steps.

By induction hypothesis, it follows that, for each \(i =1, 2\), either
\(C_i \in D(E)\) or \(\neg P\vee C_i\in D(E)\). By definition of
resolution, if both \(C_1\) and \(C_2\) are in \(D(E)\) then
\(C\in D(E)\), and in all other cases \(\neg P\vee C\in D(E)\).

As \(\bot \in D(\rho(E, P))\), it follows that either \(\bot\in D(E)\)
or \(\neg P = \neg P\vee \bot \in D(E)\). If \(\bot\in D(E)\), we have
proved unsatifiability. Otherwise we apply the same algorithm to the
restricted problem \(SAT(V \setminus \{P\}, \rho(E, \neg P))\) obtained
by assigning the value false to \(P\). In this case we deduce either
\(\bot\in D(E)\) or that \(P\in D(E)\). Thus, we either have a proof of
unsatisfiability or both \(P\in D(E)\) and \(\neg P\in D(E)\) hold. But
resolution using \(P\) and \(\neg P\) gives \(\bot\), showing that
\(\bot\in D(E)\).

The above has been sketched as an existence result, but indeed all steps
are effective, algorithmically giving a proof in the case of problems
that are not satisfiable. Note that by checking if the contradiction
requires the assumption of the truth value of \(P\) while lifting, we
have also avoided full back-tracking in some cases.

\hypertarget{looking-ahead-formal-programs-and-proofs}{%
\subsection{Looking ahead: formal programs and
proofs}\label{looking-ahead-formal-programs-and-proofs}}

Using languages implementing \emph{Dependent Type Theory} such as
\texttt{Lean\ 4} or \texttt{Idris}, one can hope for more --- a program
that outputs one of the following:

\begin{itemize}
\tightlist
\item
  a solution together with a proof that this is a solution, or
\item
  a proof that there is no solution --- not just a resolution tree but a
  proof in the given foundations.
\end{itemize}

Furthermore, the compiler should be able to verify that the program
terminates for any valid input, and has correct output of one of the
above forms. It appears that there is no obstruction in principle to
doing this, but some work is needed.
